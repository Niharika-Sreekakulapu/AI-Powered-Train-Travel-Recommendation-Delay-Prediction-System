\appendix
\section{Appendix: Reproducibility and Additional Details}
\subsection{How to reproduce the paper build}
To reproduce the figures and PDF locally:
\begin{enumerate}
  \item Create a Python virtual environment and install dependencies from `requirements.txt`.
  \item Generate figures (if model and data artifacts are present):
  \begin{verbatim}
  python paper/generate_figs.py
  \end{verbatim}
  \item Build the PDF:
  \begin{verbatim}
  cd paper
  pdflatex -interaction=nonstopmode main.tex
  bibtex main
  pdflatex -interaction=nonstopmode main.tex
  pdflatex -interaction=nonstopmode main.tex
  \end{verbatim}
  Alternatively, build the Docker image provided in `paper/Dockerfile`:
  \begin{verbatim}
  docker build -t train-delay-paper paper/
  docker run --rm -v "$PWD":/workspace train-delay-paper /bin/bash -lc "cd /workspace/paper && pdflatex -interaction=nonstopmode main.tex"
  \end{verbatim}
\end{enumerate}

\subsection{Dataset summary (demo snapshot)}
Table~\ref{tab:dataset} summarizes the demo snapshot used for figures and reporting. Note that numbers refer to the committed snapshot in `data/` and may change when using other snapshots.
\begin{table}[h]
\centering
\caption{Demo dataset snapshot summary}
\begin{tabular}{@{}lrr@{}}
\toprule
Partition & Rows & Notes \\
\midrule
Training & 6,200 & route-level aggregated records \\
Validation & 1,500 & temporally separated split \\
Test & 1,500 & held-out for reported metrics \\
\bottomrule
\end{tabular}
\label{tab:dataset}
\end{table}

\subsection{Model hyperparameters}
The demo Random Forest used in reported experiments uses a small grid; key hyperparameters are shown in Table~\ref{tab:hp}.
\begin{table}[h]
\centering
\caption{Selected hyperparameters (demo)}
\begin{tabular}{@{}lcc@{}}
\toprule
Parameter & Value & Notes \\
\midrule
n\_estimators & 100 & default tree ensemble size \\
max\_depth & 10 & early stopping for complexity \\
min\_samples\_leaf & 2 & avoid overfitting \\
random\_state & 42 & seed for reproducibility \\
\bottomrule
\end{tabular}
\label{tab:hp}
\end{table}

\subsection{Additional figures}
We include extra diagnostic figures (feature distributions, residual histograms, and calibration curves) in `paper/figs/`. These are generated by `paper/generate_figs.py` when model and data artifacts are available.

\subsection{Notes on large artifacts}
Large model and data artifacts are not committed to the main history; we recommend storing them on a release or using Git LFS for full reproducibility. See `paper/PUSH_TO_GITHUB.md` for instructions.
