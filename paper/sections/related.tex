\section{Related Work}
Ensemble methods such as Random Forests and gradient boosting are commonly used for transportation forecasting tasks due to their robustness and interpretability \cite{breiman2001random,chen2016xgboost}. Uncertainty quantification methods, including split‑conformal prediction, provide prediction intervals with finite‑sample coverage guarantees and have seen increased use in applied decision systems \cite{vovk2005algorithmic}. Explainability techniques such as SHAP help surface per‑prediction feature contributions and increase trust in model outputs \cite{lundberg2017unified}.

Domain‑specific work on train delay and reliability focuses on feature engineering at the route and temporal levels, and on combining operational data with environmental inputs such as weather. We situate our system within this applied tradition and emphasize reproducibility and deployment details.

Recent work underscores the importance of uncertainty quantification when forecasts are used in decision-making: conformal methods and conformalized quantile regression provide practical, distribution-free approaches to producing calibrated intervals, even when models are complex or nonparametric \cite{romano2019conformal}. In operations settings, uncertainty-aware recommendations have been shown to reduce downstream costs by avoiding risky routes under high epistemic uncertainty \cite{neal2022transport}.

Explainability and model introspection techniques, notably SHAP and permutation importance, are important when model outputs are exposed to end users and operators; we adopt similar attribution strategies to surface per-prediction factors that drive recommendations \cite{lundberg2017unified}.
