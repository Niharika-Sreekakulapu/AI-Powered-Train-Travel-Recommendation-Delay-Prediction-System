\section{Data and Preprocessing}
The project uses a curated route-level dataset stored in \texttt{data/} (notably \texttt{ap\_trains\_master\_clean.csv} and \texttt{train\_data\_\*.csv}). The demo dataset contains roughly 250+ route records with features including train id, source, destination, distance, day of week, month, season, weather condition, price, and historical average delay. Key preprocessing steps:
\begin{itemize}
  \item Construct \texttt{route} as \texttt{source-destination} and scale numeric features such as \texttt{distance}.
  \item Encode categorical features (\texttt{route}, \texttt{weather}, \texttt{season}) using label encoders saved with model artifacts.
  \item Apply imputation pipelines for missing RailRadar features and flag imputed values (flags v4--v7 pipeline in \texttt{scripts/}).
  \item Produce conformal calibration splits to compute 95\% prediction intervals.
\end{itemize}

Table~\ref{tab:features} summarizes key features, their types, and typical missingness in the demo snapshot.
\begin{table}[h]
\centering
\caption{Feature summary (demo snapshot)}
\begin{tabular}{@{}lcc@{}}
\toprule
Feature & Type & Typical missingness \\
\midrule
\texttt{route} & categorical & 0\% \\
\texttt{distance} & numeric & 0\% \\
\texttt{weather\_condition} & categorical & 8--12\% \\
\texttt{railradar\_speed} & numeric & 10--20\% \\
\texttt{avg\_historical\_delay} & numeric & 0\% \\
\bottomrule
\end{tabular}
\label{tab:features}
\end{table}

For reproducibility, versioned artifacts and scripts are included (e.g., \texttt{models/rr_mean_model_tuned.joblib}, \texttt{backend/model.pkl}).
