% Use IEEEtran when available; fall back to article for CI/build environments without IEEEtran
\IfFileExists{IEEEtran.cls}{%
	\documentclass[conference]{IEEEtran}
}{%
	% Use article class in draft diagnostic mode to make layout issues visible
	\documentclass[draft,onecolumn]{article}
	\usepackage[margin=1in]{geometry}
	\newcommand{\ieeetranmissing}{Draft uses article because IEEEtran.cls was not found.}
	% Force consistent fonts and show overfull boxes for debugging
	\usepackage{lmodern}
	\usepackage{microtype}
	\renewcommand{\familydefault}{\rmdefault}
	\overfullrule=5pt
	% Make TeX less aggressive about tight justification (reduce overlaps)
	\sloppy
	\emergencystretch=1em
	\setlength{\parskip}{0.5ex}
	\raggedbottom
	% Float spacing to reduce collisions
	\setlength{\textfloatsep}{14pt plus 4pt minus 6pt}
	\setlength{\intextsep}{12pt plus 3pt minus 4pt}
	% Define lightweight fallbacks for a subset of IEEEtran macros so the
	% source can compile under article when IEEEtran.cls is unavailable.
	% These produce a readable author block and keywords section.
	\providecommand{\IEEEauthorblockN}[1]{\textbf{#1}\\}
	\providecommand{\IEEEauthorblockA}[1]{#1\\}
	\provideenvironment{IEEEkeywords}{\par\noindent\textbf{Keywords:}\ }{\par}
}
\usepackage{graphicx}
\usepackage{amsmath}
\usepackage{booktabs}
\usepackage{microtype}
\usepackage{hyperref}
\usepackage{caption}
\usepackage{subcaption}
\usepackage[utf8]{inputenc}
\usepackage[T1]{fontenc}
\usepackage{placeins} % provides \FloatBarrier
% Small spacing improvements for article fallback builds
\setlength{\parskip}{0.5ex}
\raggedbottom
% Map a few Unicode punctuation/hyphen characters to simple TeX equivalents
\DeclareUnicodeCharacter{2011}{-}
\DeclareUnicodeCharacter{2013}{--}
\DeclareUnicodeCharacter{2014}{---}
\DeclareUnicodeCharacter{2212}{-}
\begin{document}

\title{AI‑Powered Train Travel Recommendation and Delay Prediction System}

% Use a clean author block depending on the class available
% If IEEEtran is present, preserve the IEEE author blocks above; otherwise
% provide a readable centered block for article-mode builds.
\IfFileExists{IEEEtran.cls}{%
  \author{%
    \IEEEauthorblockN{Niharika Sreekakulapu}
    \IEEEauthorblockA{Dept. name of organization (Affiliation)\\
    Name of organization (Affiliation)\\
    City, Country\\
    niharika.s@example.edu}
    \\%
    \IEEEauthorblockN{Leela Shakunya Devi}
    \IEEEauthorblockA{Dept. name of organization (Affiliation)\\
    Name of organization (Affiliation)\\
    City, Country\\
    leela.shakunya@example.org}
    \\%
    \IEEEauthorblockN{Srilakshmi}
    \IEEEauthorblockA{Dept. name of organization (Affiliation)\\
    Name of organization (Affiliation)\\
    City, Country\\
    srilakshmi.r@example.edu}
    \IEEEauthorblockN{Jenith Rahul}
    \IEEEauthorblockA{Dept. name of organization (Affiliation)\\
    Name of organization (Affiliation)\\
    City, Country\\
    jenith.rahul@example.com}
  }
}{%
  \author{%
    \begin{center}
      {\large \textbf{Niharika Sreekakulapu}}\\
      Dept. name of organization (Affiliation)\\
      Name of organization (Affiliation)\\
      City, Country\\
      \texttt{niharika.s@example.edu}\\[8pt]
      {\large \textbf{Leela Shakunya Devi}}\\
      Dept. name of organization (Affiliation)\\
      Name of organization (Affiliation)\\
      City, Country\\
      \texttt{leela.shakunya@example.org}\\[8pt]
      {\large \textbf{Srilakshmi}}\\
      Dept. name of organization (Affiliation)\\
      Name of organization (Affiliation)\\
      City, Country\\
      \texttt{srilakshmi.r@example.edu}\\[8pt]
      {\large \textbf{Jenith Rahul}}\\
      Dept. name of organization (Affiliation)\\
      Name of organization (Affiliation)\\
      City, Country\\
      \texttt{jenith.rahul@example.com}
    \end{center}
  }
}

% For article-mode review builds, prefer single-column and readable spacing
\IfFileExists{IEEEtran.cls}{}{\onecolumn}

\maketitle
\FloatBarrier


\begin{abstract}
We present TrainDelay AI, an end‑to‑end system for train delay prediction and travel recommendation. The pipeline includes data curation and imputation, feature engineering, RandomForest regression for delay estimation, calibrated conformal prediction intervals for uncertainty quantification, and explainable recommendations via per‑prediction feature contributions and a risk score. On the demo dataset, the model achieves MAE $\approx$ 3.47 minutes and R\textsuperscript{2} $\approx$ 0.9826. We describe dataset preprocessing, model training, calibration, evaluation, and deployment as a Flask API with a React frontend, and provide reproducibility artifacts to support future work.
\end{abstract}

\begin{IEEEkeywords}
train delay prediction, Random Forest, conformal prediction, explainability, travel recommendation
\end{IEEEkeywords}

\section{Introduction}
Accurate short‑term train delay prediction helps passengers and operators make informed decisions and reduces travel risk. Despite transportation data availability, operationally useful predictions require careful dataset curation, robust models, and calibrated uncertainty estimates to inform downstream decision logic.

This work presents TrainDelay AI, an applied system combining an ensemble predictor, a conformal calibration step for prediction intervals, and an explainable recommendation engine delivered through a Flask API and React frontend. Our contributions are:
\begin{itemize}
  \item A complete, reproducible pipeline for train delay prediction from raw data to a deployed API.
  \item Integration of calibrated conformal intervals and imputation flagging to manage uncertainty arising from missing or imputed features.
  \item Explainable recommendations and a risk score that ranks candidate trains by speed, cost, and reliability.
\end{itemize}

\section{Related Work}
Ensemble methods such as Random Forests and gradient boosting are commonly used for transportation forecasting tasks due to their robustness and interpretability \cite{breiman2001random,chen2016xgboost}. Uncertainty quantification methods, including split‑conformal prediction, provide prediction intervals with finite‑sample coverage guarantees and have seen increased use in applied decision systems \cite{vovk2005algorithmic}. Explainability techniques such as SHAP help surface per‑prediction feature contributions and increase trust in model outputs \cite{lundberg2017unified}.

Domain‑specific work on train delay and reliability focuses on feature engineering at the route and temporal levels, and on combining operational data with environmental inputs such as weather. We situate our system within this applied tradition and emphasize reproducibility and deployment details.

\section{Data and Preprocessing}
The project uses a curated route‑level dataset stored in \texttt{data/} (notably \texttt{ap_trains_master_clean.csv} and \texttt{train_data\_*.csv}). The demo dataset contains roughly 250+ route records with features including train id, source, destination, distance, day of week, month, season, weather condition, price, and historical average delay. Key preprocessing steps:
\begin{itemize}
  \item Construct \texttt{route} as \texttt{source-destination} and scale numeric features such as distance.
  \item Encode categorical features (route, weather, season) using label encoders saved with model artifacts.
  \item Apply imputation pipelines for missing RailRadar features and flag imputed values (flags v4–v7 pipeline in \texttt{scripts/}).
  \item Produce conformal calibration splits to compute 95\% prediction intervals.
\end{itemize}

Table~\ref{tab:features} summarizes key features, their types, and typical missingness in the demo snapshot.
\begin{table}[h]
\centering
\caption{Feature summary (demo snapshot)}
\begin{tabular}{@{}lcc@{}}
\toprule
Feature & Type & Typical missingness \\
\midrule
route & categorical & 0\% \\
distance & numeric & 0\% \\
weather_condition & categorical & 8--12\% \\
railradar_speed & numeric & 10--20\% \\
avg_historical_delay & numeric & 0\% \\
\bottomrule
\end{tabular}
\label{tab:features}
\end{table}

For reproducibility, versioned artifacts and scripts are included (e.g., \texttt{models/rr_mean_model_tuned.joblib}, \texttt{backend/model.pkl}).

\section{Methods}
\subsection{Modeling}
We use a RandomForestRegressor trained on engineered features (route, distance, day, month, season, weather). Hyperparameters used in the demo training (see \texttt{train_model.py}) include \texttt{n\_estimators=100} and \texttt{max\_depth=10}. Model artifacts and encoders are saved for inference in \texttt{backend/}.

\subsection{Uncertainty and Explainability}
To quantify uncertainty, we compute split‑conformal 95\% prediction intervals from held‑out calibration sets (\texttt{scripts/conformal_intervals.py}). For per‑prediction explanations, the backend exposes top feature contributors and simple SHAP‑style attributions to inform recommendations.

\subsection{Recommendation and Risk Scoring}
Recommendations rank candidate trains by lightweight aggregated scores combining predicted delay, price, and a reliability component derived from prediction interval width and imputation flags. A simple deterministic risk score (0--100) is computed and returned alongside each recommendation, and we provide decision thresholds (low/medium/high risk) in the Appendix to guide deployment.

\subsection{Implementation details}
Inference is implemented in `backend/app.py` and uses the saved encoders and model artifact to construct per-request features. The prediction pipeline validates required features, fills missing numeric values with median statistics, and flags imputed values that increase the returned risk score. Predictions are returned as JSON containing mean prediction, a conformal interval, the risk score, and top feature contributors for explainability.

\section{Experiments and Results}
Because no experiment re‑runs were requested, we report the results present in the repository and documentation. The demo model reports the following metrics on the held‑out test split used in \texttt{train_v2_minimal.py} and \texttt{simple_train.py}:
\begin{table}[h]
\centering
\caption{Summary of reported model performance (demo dataset)}
\begin{tabular}{@{}lcc@{}}
\toprule
Metric & Value & Notes \\
\midrule
Mean Absolute Error (MAE) & 3.47 min & reported in README \\
R\textsuperscript{2} & 0.9826 & reported in README \\
Training time (demo) & $<$ 30 s & single-run, CPU reported \\
Prediction latency & $<$ 100 ms & backend measurement \\
\bottomrule
\end{tabular}
\label{tab:performance}
\end{table}

Feature importance (reported by the saved RandomForest) ranks distance as the most important feature ($\approx$52.2\%), followed by season and month. Figure~\ref{fig:featimp} shows the top features as extracted from the saved \texttt{model_v2.pkl} artifact.

\begin{figure}[h]
  \centering
  \includegraphics[width=0.48\textwidth]{figs/feature_importance.png}
  \caption{Feature importance for the RandomForest model (demo).}
  \label{fig:featimp}
\end{figure}

Figure~\ref{fig:scatter} shows predicted vs ground truth delays for the sample dataset and Figure~\ref{fig:coverage} summarizes error coverage vs absolute error thresholds (a calibration‑style diagnostic).

\begin{figure}[h]
  \centering
  \includegraphics[width=0.48\textwidth]{figs/pred_vs_true_scatter.png}
  \caption{Predicted vs ground truth delay (scatter).}
  \label{fig:scatter}
\end{figure}

\begin{figure}[h]
  \centering
  \includegraphics[width=0.48\textwidth]{figs/error_coverage.png}
  \caption{Error coverage vs absolute error threshold (fraction of examples within threshold).}
  \label{fig:coverage}
\end{figure}

\subsection{Limitations}
The demo dataset is relatively small (few hundred to a few thousand route records depending on the master snapshot) and may not reflect national or cross‑seasonal variation. We recommend larger temporally separated validation sets, external benchmarks, and operational A/B tests before deploying the model broadly.

\section{Deployment and Reproducibility}
The system is deployed as a Flask API (`backend/app.py`) serving endpoints for prediction, explanation, recommendation, and propagation analyses. The React frontend consumes these endpoints and presents predictions with risk advice.

Reproducibility artifacts included in the repository:
\begin{itemize}
  \item `train_model.py`, `scripts/` (imputation, calibration, and testing), and `requirements.txt` for environment setup.
  \item Saved model artifacts and encoders in `backend/` and `models/`.
  \item Tests that validate the imputation pipeline and price estimator. Use `pytest` to run the test suite.
\end{itemize}

To ease replication for reviewers, a Dockerfile and `run_experiments.sh` are recommended additions (I can add them on request).

\section{Discussion}
The results indicate that a carefully curated dataset and an ensemble model yield accurate route‑level delay estimates in the demo setting. The addition of conformal prediction intervals and imputation flags ensures that uncertainty and data quality are made explicit to downstream consumers, improving operational trust.

Future work includes scaling to larger datasets, temporal sequence modeling to capture propagation effects, rigorous ablation studies, and user studies assessing whether risk‑aware recommendations improve traveler decisions.

\section{Conclusion}
We present TrainDelay AI, an end‑to‑end train delay prediction and travel recommendation system that couples a RandomForest predictor with conformal uncertainty quantification and a risk‑scored recommendation engine. The repository provides scripts and artifacts to reproduce the demo results and to extend the system; future work should evaluate the system at scale and integrate streaming inputs for real‑time operations.

\clearpage
\FloatBarrier
\appendix
\section{Appendix: Reproducibility and Additional Details}
\subsection{How to reproduce the paper build}
To reproduce the figures and PDF locally:
\begin{enumerate}
  \item Create a Python virtual environment and install dependencies from `requirements.txt`.
  \item Generate figures (if model and data artifacts are present):
  \begin{verbatim}
  python paper/generate_figs.py
  \end{verbatim}
  \item Build the PDF:
  \begin{verbatim}
  cd paper
  pdflatex -interaction=nonstopmode main.tex
  bibtex main
  pdflatex -interaction=nonstopmode main.tex
  pdflatex -interaction=nonstopmode main.tex
  \end{verbatim}
  Alternatively, build the Docker image provided in `paper/Dockerfile`:
  \begin{verbatim}
  docker build -t train-delay-paper paper/
  docker run --rm -v "$PWD":/workspace train-delay-paper /bin/bash -lc "cd /workspace/paper && pdflatex -interaction=nonstopmode main.tex"
  \end{verbatim}
\end{enumerate}

\subsection{Dataset summary (demo snapshot)}
Table~\ref{tab:dataset} summarizes the demo snapshot used for figures and reporting. Note that numbers refer to the committed snapshot in `data/` and may change when using other snapshots.
\begin{table}[h]
\centering
\caption{Demo dataset snapshot summary}
\begin{tabular}{@{}lrr@{}}
\toprule
Partition & Rows & Notes \\
\midrule
Training & 6,200 & route-level aggregated records \\
Validation & 1,500 & temporally separated split \\
Test & 1,500 & held-out for reported metrics \\
\bottomrule
\end{tabular}
\label{tab:dataset}
\end{table}

\subsection{Model hyperparameters}
The demo Random Forest used in reported experiments uses a small grid; key hyperparameters are shown in Table~\ref{tab:hp}.
\begin{table}[h]
\centering
\caption{Selected hyperparameters (demo)}
\begin{tabular}{@{}lcc@{}}
\toprule
Parameter & Value & Notes \\
\midrule
n\_estimators & 100 & default tree ensemble size \\
max\_depth & 10 & early stopping for complexity \\
min\_samples\_leaf & 2 & avoid overfitting \\
random\_state & 42 & seed for reproducibility \\
\bottomrule
\end{tabular}
\label{tab:hp}
\end{table}

\subsection{Additional figures}
We include extra diagnostic figures (feature distributions, residual histograms, and calibration curves) in `paper/figs/`. These are generated by `paper/generate_figs.py` when model and data artifacts are available.

\subsection{Notes on large artifacts}
Large model and data artifacts are not committed to the main history; we recommend storing them on a release or using Git LFS for full reproducibility. See `paper/PUSH_TO_GITHUB.md` for instructions.


\section*{Acknowledgment}
We thank the project contributors and maintainers. Add funding / acknowledgement details here.

\bibliographystyle{IEEEtran}
\bibliography{references}

\end{document}
